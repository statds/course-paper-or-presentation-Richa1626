\documentclass[12pt]{article}

%% preamble: Keep it clean; only include those you need
\usepackage{amsmath}
\usepackage[margin = 1in]{geometry}
\usepackage{graphicx}
\usepackage{booktabs}
\usepackage{natbib}

% for space filling
\usepackage{lipsum}
% highlighting hyper links
\usepackage[colorlinks=true, citecolor=blue]{hyperref}

\title{The Effect of Airbnb on Hotel Performance: Comparing Single- and Multi-Unit Host Listings in the United States }
\author{Richa Patel\\
Department of Statistics\\
University of Connecticut
}
\date{2023-10-26}

\begin{document}

\maketitle 

\paragraph{Abstract}
The sharing economy, exemplified by platforms like Airbnb, has reshaped the accommodations sector, potentially influencing traditional lodging establishments. The study focuses on multi-unit host listings and investigates how Airbnb affects hotel income negatively. It implies that price cuts, not a downturn in demand, are the main cause of the loss in hotel income. The pricing influence of single-unit host listings on hotels is greater than that of multi-unit listings. This effect is seen in a variety of hotel segments, with pressure to lower prices and substitution effects for economy-scale hotels. 
\end{abstract}

\Paragraph{Objective}

\begin{itemize}

\item To quantify the impact of Airbnb listings (both single-unit and multi-unit) on hotel performance.
\item Identify key performance indicators affected by Airbnb's presence.

\end{itemize}

\praragraph{Theoritical Background}

\subparagraph{P2P Accomodation}
The sharing economy, which combines traditional market exchanges with access-based goods or services, has rapidly spread across various sectors, including the accommodation sector. P2P accommodation units, which are space rented for short-term stays using digital platforms, have transformed the industry. Research in tourism and hospitality management has focused on understanding guests' motivations, determinants of P2P accommodation demand, hosts' behavioral patterns, impact on international tourism demand, and societal and economic impacts. Factors such as platform-mediated transactions, necessary monetary exchange, and the broad scope of providers have hastened the spread of these networks.

\subparagraph{Multi-Unit Hosting}
Multi-unit hosting refers to hosts on platforms like Airbnb who offer multiple properties or units for short-term rental. These hosts may own or manage multiple apartments, houses, or other types of accommodations, and rent them out to travelers. They typically do not reside in the units themselves, and the properties are primarily intended for rental purposes.

\subparagraph{Single-Unit Hosting}
Single-unit hosting, on the other hand, refers to hosts who rent out a single property or unit on platforms like Airbnb. In this case, the host is typically the owner or primary resident of the property being rented out. This type of hosting often involves renting out one's own primary residence, which means the space may contain personal belongings and reflect the host's individual style and preferences. Guests may encounter items like clothing, family pictures, and food in the pantry, indicating that they are staying in someone else's home.

\paragraph{Data}

\begin{itemize}

\item Independent Variable is used to see its effect on dependent variables.In this case the study looks at how different types of Airbnb listings impact the performance of hotels across various U.S. states. It covers all states except Delaware, and includes Washington D.C. in its analysis.

\item The dependent variable is the main factor or outcome that the study aims to understand or explain. In this case, it is related to measures of hotel performance. Common metrics of hotel performance mentioned are:

\begin{enumerate}

 \item Average Daily Rate (ADR): The average amount of money earned by a hotel per room per day.
 \item Occupancy Rate: The percentage of available rooms that are actually occupied.
 \item Revenue per Available Room (RevPAR): A performance metric that combines occupancy rates and average daily rates to provide a comprehensive measure of hotel revenue.

\end{enumerate}
\end{itemize}

\paragraph{Methodology}

\subparagraph{Regression Analysis}
Utilize multiple regression models to investigate the relationship between Airbnb listings (both single- and multi-unit) and hotel performance metrics. And Incorporate control variables such as location, seasonality, and hotel amenities.

\subparagraph {Comparative Analysis}
Compare the impacts of single-unit and multi-unit host listings on hotel performance measures to identify any significant disparities.

\paragraph {Data Analysis}

\subparagraph{Descriptive Statistics}

Provide an overview of the distribution of Airbnb and hotel listings, occupancy rates, and customer reviews.
Analyze summary statistics by host type and state.

\subparagraph{Regression Models}

Perform separate regression analyses for single-unit and multi-unit host listings to understand their respective impacts on RevPAR, ADR, and OCC.
Assess model fit and robustness through diagnostic tests.


\paragraph{Results}

\begin{itemize}

\item Determine the extent to which Airbnb listings influence hotel performance metrics.
\item Identify any differential effects between single- and multi-unit host listings.
\item Analyze if the relationship between Airbnb and hotel performance varies across different U.S. states.

\end{itemize}

\paragraph{Conclusion}

This proposed statistical study aims to contribute significant insights into the evolving relationship between Airbnb and traditional lodging establishments, with a specific focus on host type differentiation. Through rigorous statistical analysis, we endeavor to provide empirical evidence to inform strategic decision-making within the hospitality industry.

\bibliography{ref}
\bibliographystyle{plain}
\end{document}

\end{document}
