\documentclass[12pt]{article}
\usepackage{amsmath}
\usepackage[margin = 1in]{geometry}
\usepackage{graphicx}
\usepackage{booktabs}
\usepackage{natbib}
\usepackage{setspace}
\usepackage{float}
\usepackage{lipsum}
\usepackage[colorlinks=true, citecolor=blue]{hyperref}


\title{Factors Affecting Movies Gross Revenue}

\author{Richa Patel\\
  Jun Yan\\[2ex]
  Department of Statistics\\
  University of Connecticut\\
}

\begin{document}

\maketitle
\doublespace

\begin{abstract}

The expanding economy and changing cultural expectations are driving the film industry's 
steady expansion. This study applies knowledge from previous studies to examine factors 
that impact box office income. The study examines important factors such income, popularity,
runtime, and vote average using a dataset that was obtained from the Kaggle website and 
included information on over 6820 movies. Techniques for visual analysis are used to 
comprehend how they are interdependent. The results demonstrate that three important 
factors influence a film's financial performance at the box office: popularity, budget,
and vote average. The statistically substantial correlations between them are shown using 
linear regression modeling. This study offers useful insights into the elements that 
contribute to a film's box office success, direction for stakeholders, and doable
recommendations to improve each specific film's performance. The report, which is supported 
by an in-depth understanding of the variables influencing box office revenue, presents
a positive assessment of the film industry's future course.


\bigskip
\noindent{\sc Keywords}:
Linear Regression; 
Factors;
Movie Revenue;
Data Analysis;
Key Factors of Movie Sucess;
Impact on Revenue;

\end{abstract}


\section{Introduction}
\label{sec:intro}

Since the dawn of cinema, a lot has evolved in terms of effects and sound design. 
Silent or black-and-white films are no longer common. However, there had been no
change in the variables that are used to predict revenues. The stars, directors,
authors, budget, corporation, rating, and scores are among those elements. With 
varying ratings and genres, movies are meant to amuse their viewers.Since IMDB is the
most popular website for movie reviews and ratings, it could be interesting to look at 
user reviews and find out what people liked and didn't like, which might help us determine
whether or not the movie has satisfied viewers, which could have a positive or negative 
effect on the movie's box office performance \citep{3}. However, the writers' and directors' perspectives on
those films, together with the actors whose acting prowess transforms the director's vision 
into a live action, are what stimulates their interest in those films the most. Sometimes
stars with a lot of popularity may also have a big impact on the direction of movies. In certain cases,
obtaining agreements from distributors, exhibitors, producers, investors, and sponsors 
even serves as the "green light" \citep{2}. That being said, movies deliver their 
viewers those things. The most significant measure of a movie's popularity among viewers
is its box office earnings, Similarly, the best measure of a film's earnings is its box 
office performance \citep{1}. The gross revenue, however, is what really matters most as it 
determines how well the film performed at the box office.

A similar study on the factors determining revenue was carried out in 2023 by Bingyu Hao 
of York University in Canada. Hao's study concentrated on the TMDb Movies dataset, which 
contains details about more than 10,000 films and their attributes including budget, runtime, 
revenue, popularity, and vote average, among others. He made use of the revenue distribution 
and scatter graph in relation to budget, runtime, vote average, and popularity, among other 
factors for EDA analysis. Then, he employed fitted regression and linear regression as his models. Then ultimately
came to the conclusion that the last factors influencing the ability to anticipate movie revenue 
are popularity, vote average, and budget.

The data set in this study provides details on the variable found in Section~\ref{sec:data}.  
It is followed by a brief explanation of the methodology used in Section~\ref{sec:meth}, 
which describes the study that was conducted to anticipate the variables influencing movie 
income. Results are presented in Section~\ref{sec:res}, which also includes a graphic representation
explaining the analysis's conclusions. Finally, there is the discussion section, 
Section~\ref{sec:dis}, which contains the conclusion and further expectations for this
paper's consideration.

\section{Data}
\label{sec:data}

The data of this paper is taken from Kaggle \href{https://www.kaggle.com/datasets/danielgrijalvas/movies}. 
The data includes 6820 films from 1986 to 2016 (about 220 films annually). It has
fifteen features: budget, runtime (the amount of time the film is on screen), company 
or productions, country, director, genre, gross (the amount of money the film brings in), 
name of the film, rating, date of release, IMDb ratings, votes, actors starring in the
film, year of release, and writers. I will be analyzing the variables that affect
gross revenue in this paper by concentrating on the directors, stars, scores,votes and budget.

\section{Methods}
\label{sec:meth}

After selecting the 5 varibles to focus on gross revenue of movies, I would be using
liner regression model an my analysis medthod. But Before getting started into detailed 
regression model, I converted the nomial columns to numerical columns. As the data contained 
numerical data that is score, budget, votes, runtime and nominal data that are directors, 
stars, writers, ratings, year of realease, genre, company, country. Afterwards, I looked 
for any missing values or NA values and replaced them with 0.

For EDA analysis as it is to gaining insights and understanding the characteristics 
of the data I will be using Correlation graph.Allowing to identify patterns, dependencies,
or potential relationships among variables.

\begin{figure}
  \centering
	\includegraphics[width=1\textwidth]{Correlationgraph.pdf}
	\label{fig:Correlationgraph}
In Figure~\ref{fig:Correlationgraph}, the plots show the corelation graph.
\end{figure}

The correlation between each axis’s variables is displayed by each square. From 
-1 to +1 is the correlation range. A linear trend between the two variables is shown
by values that are closer to zero. Positive correlation indicates that the closer one 
is to 1, the greater the link; that is, as one grows, so does the other. When one variable 
decreases while the other grows, the correlation is comparable when it is closer to -1.
Because the squares there are perfectly correlated, each variable is correlated with itself,
which is why all of the diagonals are 1/dark blue. Regarding the other variables, a greater 
connection is indicated by a larger number and a darker hue. The plot is also symmetrical
about the diagonal since the same two variables are being paired together in those squares.

The data was then divided into training and test sets so that a model could be 
trained on one set and its performance assessed on the other, with a 7:3 split 
ratio adjustment made. And then Linear regression is applied to 
the data set.

The model of the linear regression is:
$Revenue = \beta_0 +\beta_1*directors + \beta_2*stars + \beta_3*scores + \beta_4*votes \beta_5*budget$.

\section{Results}
\label{sec:res}

\begin{table}[h]
\caption{Table 1: Linear Regression Model}
\centering
\begin{tabular}{rrrrr}
\hline
 & Estimate   & Std.Error & t value & Pr (\textgreater|t|) \\
 \hline
intercept & -4.745e+07 & 9.255e+06 & -5.127  & 3.05e-07 \\ 
director  & 9.393e+03  & 2.153e+03 & 4.363   & 1.31e-05 \\ 
star      & -3.136e+02 & 2.245e+03 & -0.140  & 0.8889 \\ 
score     & 3.391e+06  & 1.441e+06 & 2.353   & 0.0187 \\ 
votes     & 3.178e+02  & 9.622e+00 & 33.028  & 2e-16 \\ 
budget    & 2.501e+00  & 3.848e-02 & 65.010  & 2e-16 \\
\hline
\end{tabular}
\end{table}

The linear regression model's coefficients are displayed in Table 1 for our observation. 
Additionally, each variable's p-value is has to beless than 0.05 in order to be significantly
inpacting Revenue. The data indicates that neither the variable star (p-value of 0.8886) 
nor the score (p-value of 0.0186) are statistically significant in predicting the result.
Not as significant as other factors, even if the score's p-value is 0.0186 which is somewhat
below the usual significance level of 0.05. 

\begin{table}[h]
\caption{Table 2: Fitted Linear Regresion Model}
\centering
\begin{tabular}{rrrrr}
\hline
 & Estimate   & Std.Error & t value & Pr (\textgreater|t|) \\ 
\hline
(Intercept) & -2.649e+07 & 2.312e+06 & -11.454 & $<$ 2e-16 \\
director    & 9.300e+03  & 1.568e+03 & 5.931   & 3.2e-09 \\
votes       & 3.276e+02  & 8.675e+00 & 37.765  & $<$ 2e-16 \\
budget      & 2.486e+00  & 3.791e-02 & 65.576  & $<$ 2e-16 \\
\hline
\end{tabular}
\end{table}

The factors star and score were eliminated because of greater p-values than 0.05, 
which indicated that they had little to no impact on revenue. In order to have a 
more thorough understanding of the variables influencing revenue, an altered linear 
regression model was developed. Table 2 shows a p-value of less than 0.05 for each 
of the variables—director, votes, and budget—indicating that these factors have a
considerable influence on revenue.

The budget, vote average, and popularity are significant independent variables with
p-values less than 0.05, according to Hao's prediction, which has a significance p-value
of 0.05. Therefore, in order to analyze movie earnings, he looked at votes, budget, and 
population. The variables that were utilized are identical to the ones I had used for my 
analysis, and after the fitted regression, the variables with p-values less than 0.05 are
also similar. However, I used directors as my variable rather than pupulation.

Final Model: 
$Revenue = -2.649e+07 +9300*directors + 327.6*votes + 2.486*budget$.


As in this model, directors, budget and votes are the most influential variables.
Since the coefficient of directors is 9300, it may be inferred that selecting a 
director for a unit boosts revenue by 9300. Comparably, the votes coefficent of 
327.6 means that a unit increase in votes corresponds to an indirect gain in income 
of 237.6. The budget's coefficient of 2.486 indicates that a unit increase in the 
budget will result in a 2.486 increase in revenue.



\section{Discussion}
\label{sec:dis}



\bibliography{refs}
\bibliographystyle{plain}
\end{document}

