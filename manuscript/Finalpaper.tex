\documentclass[12pt]{article}
\usepackage{amsmath}
\usepackage[margin = 1in]{geometry}
\usepackage{graphicx}
\usepackage{booktabs}
\usepackage{natbib}
\usepackage{setspace}
\usepackage{float}
\usepackage{lipsum}
\usepackage[colorlinks=true, citecolor=blue]{hyperref}


\title{Factors Affecting Movies Gross Revenue}

\author{Richa Patel\\
  Jun Yan\\[2ex]
  Department of Statistics\\
  University of Connecticut\\
}

\begin{document}

\maketitle
\doublespace

\begin{abstract}


\bigskip
\noindent{\sc Keywords}:
  

\end{abstract}


\section{Introduction}
\label{sec:intro}

Since the dawn of cinema, a lot has evolved in terms of effects and sound design. 
Silent or black-and-white films are no longer common. However, there had been no
change in the variables that are used to predict revenues. The stars, directors,
authors, budget, corporation, rating, and scores are among those elements. With 
varying ratings and genres, movies are meant to amuse their viewers. However, 
the writers' and directors' perspectives on those films, together with the actors 
whose acting prowess transforms the director's vision into a live action, are what 
stimulates their interest in those films the most. Sometimes stars with a lot of 
popularity may also have a big impact on the direction of movies. In certain cases,
obtaining agreements from distributors, exhibitors, producers, investors, and sponsors 
even serves as the "green light" \citep{4}. That being said, movies deliver their 
viewers those things. The most significant measure of a movie's popularity among viewers
is its box office earnings, Similarly, the best measure of a film's earnings is its box 
office performance \citep{1}. The gross revenue, however, is what really matters most as it 
determines how well the film performed at the box office.

The data set in this study provides details on the variable found in Section~\ref{sec:data}. 
It is followed by a brief explanation of the methodology used in Section~\ref{sec:meth}, 
which describes the study that was conducted to anticipate the variables influencing movie 
income. Results are presented in Section~\ref{sec:res}, which also includes a graphic representation
explaining the analysis's conclusions. Finally, there is the discussion section, 
Section~\ref{sec:dis}, which contains the conclusion and further expectations for this
paper's consideration.

Similar research on the elements influencing revenue  was carried out in 2023 by Bingyu Hao 
of York University in Canada. The research focused on revenue, popularity, budget, runtime, and vote average.

\section{Data}
\label{sec:data}

The data of this paper is taken from Kaggle \href{https://www.kaggle.com/datasets/danielgrijalvas/movies}. 
The data includes 6820 films from 1986 to 2016 (about 220 films annually). It has
fifteen features: budget, runtime (the amount of time the film is on screen), company 
or productions, country, director, genre, gross (the amount of money the film brings in), 
name of the film, rating, date of release, IMDb ratings, votes, actors starring in the
film, year of release, and writers. I will be analyzing the variables that affect
gross revenue in this paper by concentrating on the directors, stars, scores,votes and budget.

\section{Methods}
\label{sec:meth}

After selecting the 5 varibles to focus on gross revenue of movies, I would be using
liner regression model an my analysis medthod. But Before getting started into detailed 
regression model, I converted the nomial columns to numerical columns. As the data contained 
numerical data that is score, budget, votes, runtime and nominal data that are directors, 
stars, writers, ratings, year of realease, genre, company, country. Afterwards, I looked for any missing values or NA values and replaced them with 0.

For EDA analysis as it is to gaining insights and understanding the characteristics of the data I will be using Correlation graph.Allowing to identify patterns, dependencies, or potential relationships among variables. The data was then divided into training and test sets so that a model could be trained on one set and its performance assessed on the other, with a 7:3 split ratio adjustment made. And then Linear regression is applied to the data set.


The model of the linear regression is:
$Revenue = \beta_0 +\beta_1*directors + \beta_2*stars + \beta_3*scores + \beta_4*votes \beta_5*budget$.

\begin{figure}
  \centering
	\caption{Correlation Graph}
	\includegraphics[width=1\textwidth]{Correlationgraph.pdf}
	\label{fig:Correlationgraph}
In Figure~\ref{fig:Correlationgraph}, the plots show the corelation graph.
\end{figure}

\section{Results}
\label{sec:res}

\begin{table}[h]
\centering
\caption{Table 1: Linear Regression Model}
\begin{tabular}{|l|c|c|c|c|}
\hline
 & Estimate   & Std.Error & t value & Pr (\textgreater{}|t|) \\ 
 \hline
intercept & -4.745e+07 & 9.255e+06 & -5.127  & 3.05e-07 \\ 
director  & 9.393e+03  & 2.153e+03 & 4.363   & 1.31e-05 \\ 
star      & -3.136e+02 & 2.245e+03 & -0.140  & 0.8889 \\ 
score     & 3.391e+06  & 1.441e+06 & 2.353   & 0.0187 \\ 
votes     & 3.178e+02  & 9.622e+00 & 33.028  & 2e-16 \\ 
budget    & 2.501e+00  & 3.848e-02 & 65.010  & 2e-16 \\
\hline
\end{tabular}
\end{table}

The linear regression model's coefficients are displayed in Table 1 for our observation. Additionally, each variable's p-value is has to beless than 0.05 in order to be significantly inpacting Revenue. The data indicates that neither the variable star (p-value of 0.8886) nor the score (p-value of 0.0186) are statistically significant in predicting the result. Not as significant as other factors, even if the score's p-value is 0.0186 which is somewhat below the usual significance level of 0.05.

\begin{table}[h]
\centering
\caption{Table 2: Fitted Linear Regresion Model}
\begin{tabular}{|l|c|c|c|c|}
\hline
 & Estimate   & Std.Error & t value & Pr (\textgreater{}|t|) \\ 
\hline
(Intercept) & -2.649e+07 & 2.312e+06 & -11.454 & $<$ 2e-16 \\
director    & 9.300e+03  & 1.568e+03 & 5.931   & 3.2e-09 \\
votes       & 3.276e+02  & 8.675e+00 & 37.765  & $<$ 2e-16 \\
budget      & 2.486e+00  & 3.791e-02 & 65.576  & $<$ 2e-16 \\
\hline
\end{tabular}
\end{table}

The factors star and score were eliminated because of greater p-values than 0.05, which indicated that they had little to no impact on revenue. In order to have a more thorough understanding of the variables influencing revenue, an altered linear regression model was developed. Table 2 shows a p-value of less than 0.05 for each of the variables—director, votes, and budget—indicating that these factors have a considerable influence on revenue.


\section{Discussion}
\label{sec:dis}



\bibliography{refs}
\bibliographystyle{plain}
\end{document}

